\documentclass[11pt, fontset=windows, ignorenonframetext]{beamer}  % ignorenonframetext 忽略任何 frame 之外的所有文本

\usepackage[backend=bibtex]{biblatex}
\usepackage[noindent]{ctex} % 中文
\usepackage[T1]{fontenc} %  风格数学字体
\usepackage{hyperref} % 超链接命令
\usepackage{graphicx} % 支持插图
\usepackage{subfigure} % 子图片
\usepackage{times} % 使用 Times New Roman 字体
\usepackage{bookmark}
\usepackage{booktabs} % 三线表
\usepackage{latexsym} % 数学符号环境
\usepackage{amsmath,amsfonts,amssymb,amscd,amsthm,calligra,mathrsfs}
\usepackage{xcolor,comment,pstricks,listings,stackengine}
\usepackage{multicol,multirow} % 多栏
\usepackage{array} % 表格
\usepackage{ragged2e} % 段落对齐
\usepackage[english]{babel}
\usepackage{enumerate} % 列表环境
\usepackage{bm} % 提供将数学符号加粗的命令 \bm
\usepackage{csquotes} % 引号
\usepackage{fontspec} % 设置字体



%-------- theme --------
\usetheme{Madrid}
\usecolortheme{default}
\useinnertheme{circles}
\setbeamertemplate{title page}[default][colsep=-4bp,rounded=true]  % 去掉 block 的阴影
% no navigation icons:
\usenavigationsymbolstemplate{}

%--------------- color --------------------
\definecolor{ustgold}{RGB}{153, 102, 0}
\definecolor{ustblue}{RGB}{0, 51, 102}

\definecolor{Logo1}{rgb}{0.208, 0.2865, 0.373}
\definecolor{Logo2}{rgb}{0.000, 0.674, 0.863}

\setbeamercolor*{palette primary}{bg=ustblue, fg=white}
\setbeamercolor*{palette secondary}{bg=ustblue, fg=white}
\setbeamercolor*{palette tertiary}{bg=white, fg=ustblue}

\setbeamercolor{title}{fg=ustblue,bg=white}         % Title of presentation
\setbeamercolor{section title}{fg=ustblue,bg=white}
\setbeamercolor{subsection title}{fg=ustblue, bg=white}
\setbeamercolor{frametitle}{fg=ustblue,bg=white}
% set the blocks to match the beaver colors:
\setbeamercolor{structure}{fg=ustblue,bg=white}
% inside the blocks to grey:
\setbeamercolor{block body}{bg=gray!10!white}
%------------------------------------------------------------


%--------------- START PREAMBLE ---------------
%This block of code defines the information to appear in the title page

\title[]{The amazing title of your presentation}

\subtitle{And its subtitle}

\author[Iura Arata] % (optional)
{Iura Arata}

\institute[HKUST(GZ)] % (optional)
{
FinTech Thrust, Society Hub\\[0.3em]
The Hong Kong University of Science and Technology (Guangzhou)
}

\date[Sep 2022] % (optional)
{Sep 2022}

\logo{\includegraphics[height=.5cm]{./Figure/UST-GZ-logo-2.pdf}}

%End of title page configuration block
%------------------------------------------------------------


%------------------------------------------------------------
%The next block of commands puts the table of contents at the 
%beginning of each section and highlights the current section:

\AtBeginSection[]
{
\begin{frame}
    \frametitle{Table of Contents}
    \tableofcontents[currentsection]
\end{frame}
}
%------------------------------------------------------------


\begin{document}

\addtocounter{framenumber}{-1}

%The next statement creates the title page.
\frame{\titlepage}


%---------------------------------------------------------
%This block of code is for the table of contents after
%the title page
\begin{frame}
\frametitle{Table of Contents}
\tableofcontents
\end{frame}
%---------------------------------------------------------


\section{First section}

%---------------------------------------------------------
%Changing visivility of the text
\begin{frame}
\frametitle{Sample frame title}
This is a text in second frame. For the sake of showing an example.

\begin{itemize}
    \item<1-> Text visible on slide 1
    \item<2-> Text visible on slide 2
    \item<3> Text visible on slides 3
    \item<4-> Text visible on slide 4
\end{itemize}
\end{frame}

%---------------------------------------------------------


%---------------------------------------------------------
%Example of the \pause command
\begin{frame}
In this slide \pause

the text will be partially visible \pause

And finally everything will be there
\end{frame}
%---------------------------------------------------------

\section{Second section}

%---------------------------------------------------------
%Highlighting text
\begin{frame}
\frametitle{Sample frame title}

In this slide, some important text will be
\alert{highlighted} because it's important.
Please, don't abuse it.

\begin{block}{Remark}
Sample text
\end{block}

\begin{alertblock}{Important theorem}
Sample text in red box
\end{alertblock}

\begin{examples}
Sample text in green box. The title of the block is ``Examples".
\end{examples}
\end{frame}
%---------------------------------------------------------


%---------------------------------------------------------
%Two columns
\begin{frame}
\frametitle{Two-column slide}

\begin{columns}

\column{0.5\textwidth}
This is a text in first column.
$$E=mc^2$$
\begin{itemize}
\item First item
\item Second item
\end{itemize}

\column{0.5\textwidth}
This text will be in the second column
and on a second tought this is a nice looking
layout in some cases.
\end{columns}
\end{frame}
%---------------------------------------------------------


\end{document}